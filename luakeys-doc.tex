\documentclass{ltxdoc}

\usepackage{hyperref}
\EnableCrossrefs
\CodelineIndex
\RecordChanges

\usepackage{mdframed}
\usepackage{minted}
\usepackage{luakeys-debug}
\usepackage{multicol}
\usepackage{luacode}
\usepackage{syntax}
\usemintedstyle{friendly}
\BeforeBeginEnvironment{minted}{\begin{mdframed}}
\AfterEndEnvironment{minted}{\end{mdframed}}
\setminted{
  breaklines=true,
  fontsize=\footnotesize,
}
\def\lua#1{\mintinline{lua}|#1|}

\begin{document}

\providecommand*{\url}{\texttt}

\title{The \textsf{luakeys} package}
\author{%
  Josef Friedrich\\%
  \url{josef@friedrich.rocks}\\%
  \href{https://github.com/Josef-Friedrich/luakeys}{github.com/Josef-Friedrich/luakeys}%
}
\date{v0.5 from 2022/04/04}

\maketitle

\vfill

%level1={level2={level3={dim=1cm,bool=true,num=-0.001,str=lua}}}

\begin{minted}{lua}
local luakeys = require('luakeys')
local kv = luakeys.parse('level1={level2={level3={dim=1cm,bool=true,num=-0.001,str=lua}}}')
luakeys.print(kv)
\end{minted}

\noindent
Result:

\begin{center}
\begin{minted}{lua}
{
  ['level1'] = {
    ['level2'] = {
      ['level3'] = {
        ['dim'] = 1864679,
        ['bool'] = true,
        ['num'] = -0.001
        ['str'] = 'lua',
      }
    }
  }
}
\end{minted}
\end{center}

\vfill

\strut

\newpage

\tableofcontents

\newpage

\section{Introduction}

\noindent
|luakeys| is a Lua module that can parse key-value options like the
\TeX{} packages \href{https://www.ctan.org/pkg/keyval}{keyval},
\href{https://www.ctan.org/pkg/kvsetkeys}{kvsetkeys},
\href{https://www.ctan.org/pkg/kvoptions}{kvoptions},
\href{https://www.ctan.org/pkg/xkeyval}{xkeyval},
\href{https://www.ctan.org/pkg/pgfkeys}{pgfkeys} etc. do. |luakeys|,
however, accomplishes this task entirely, by using the Lua language and
doesn’t rely on \TeX{}. Therefore this package can only be used with the
\TeX{} engine Lua\TeX{}. Since |luakeys| uses
\href{http://www.inf.puc-rio.br/~roberto/lpeg/}{LPeg}, the parsing
mechanism should be pretty robust.

The TUGboat article
\href{http://www.tug.org/tugboat/tb30-1/tb94wright-keyval.pdf}
{“Implementing key–value input: An introduction” (Volume 30 (2009), No. 1)}
by Joseph Wright and Christian Feuersänger gives a good overview of the
available key-value packages.

This package would not be possible without the article
\href{https://tug.org/TUGboat/tb40-2/tb125menke-lpeg.pdf}
{“Parsing complex data formats in LuaTEX with LPEG” (Volume 40 (2019), No. 2)}.

%-----------------------------------------------------------------------
%
%-----------------------------------------------------------------------

\clearpage

\section{Usage}

%%
%
%%

\subsection{Using the Lua module \texttt{luakeys.lua}}

The core functionality of this package is realized in Lua. So you can
use \texttt{luakeys} without using the wrapper \TeX{} files
\texttt{luakeys.sty} and \texttt{luakeys.tex}.

\begin{minted}{latex}
\documentclass{article}
\directlua{
  luakeys = require('luakeys')
}

\newcommand{\helloworld}[2][]{
  \directlua{
    local keys = luakeys.parse('\luaescapestring{\unexpanded{#1}}')
    luakeys.print(keys)
    local marg = '#2'
    tex.print(keys.greeting .. ', ' .. marg .. keys.punctuation)
  }
}
\begin{document}
\helloworld[greeting=hello,punctuation=!]{world}
\end{document}
\end{minted}

%%
%
%%

\subsection{Using the Lua\LaTeX{} wrapper \texttt{luakeys.sty}}

The supplied Lua\LaTeX{} file is quite small:

\begin{minted}{latex}
\NeedsTeXFormat{LaTeX2e}
\ProvidesPackage{luakeys}
\directlua{luakeys = require('luakeys')}
\end{minted}

\noindent
It loads the Lua module into the global variable \texttt{luakeys}.

\begin{minted}{latex}
\documentclass{article}
\usepackage{luakeys}

\begin{document}
  \directlua{
    local keys = luakeys.parse('one,two,three')
    tex.print(keys[1])
    tex.print(keys[2])
    tex.print(keys[3])
  }
\end{document}
\end{minted}

%%
%
%%

\subsection{Using the plain Lua\TeX{} wrapper \texttt{luakeys.tex}}

Even smaller is the file \texttt{luakeys.tex}. It consists of only one
line:

\begin{minted}{latex}
\directlua{luakeys = require('luakeys')}
\end{minted}

\noindent
It does the same as the Lua\LaTeX{} wrapper and loads the Lua module
\texttt{luakeys.lua} into the global variable \texttt{luakeys}.

\begin{minted}{latex}
\input luakeys.tex

\directlua{
  local keys = luakeys.parse('one,two,three')
  tex.print(keys[1])
  tex.print(keys[2])
  tex.print(keys[3])
}
\bye
\end{minted}

\section{Exported functions of the Lua module \texttt{luakeys.lua}}

To learn more about the individual functions (local functions), please
read the \href{https://josef-friedrich.github.io/luakeys/}{source code
documentation}, which was created with
\href{http://stevedonovan.github.io/ldoc/}{LDoc}. The Lua module exports
this functions:

\begin{minted}{lua}
local luakeys = require('luakeys')
local parse = luakeys.parse
local render = luakeys.render
--local print = luakeys.print -- That would overwrite the built-in Lua function
local save = luakeys.save
local get = luakeys.get
\end{minted}

%%
%
%%

\subsection{\texttt{parse(kv\_string, opts)}: table}
\label{parse}

The function \lua{parse(kv_string, opts)} parses a key-value string
into a Lua table.

\begin{minted}{latex}
\newcommand{\mykeyvalcmd}[2][]{
  \directlua{
    result = luakeys.parse('#1')
    luakeys.print(result)
  }
  #2
}
\mykeyvalcmd[one=1]{test}
\end{minted}

\noindent
In plain \TeX:

\begin{minted}{latex}
\def\mykeyvalcommand#1{
  \directlua{
    result = luakeys.parse('#1')
    luakeys.print(result)
  }
}
\mykeyvalcmd{one=1}
\end{minted}

\noindent
The function can be called with an options table. This options are
supported:

\begin{minted}{lua}
local opts = {
  -- { KEY = 'Value' } -> { key = 'value' }
  case_insensitive_keys = false,

  -- Visit all key-value pairs in the recursive parse tree.
  converter = function(key, value, depth, current_tree, root_tree)
    return key, value
  end,

  -- Automatically convert dimensions into scaled points (1cm -> 1864679).
  -- default: true
  convert_dimensions = false,

  -- Output the results table to the console.
  -- default: false
  debug = true,

  -- A table with some default values. The result table is merged with
  -- this table.
  defaults = { key = 'value' },

  -- Key-value pair defintions.
  definitions = defs,

  -- Standalone values aka values with numeric keys are converted to
  -- keys holding the value true:
  -- { 'one', 'two', 'three' } -> { one = true, two = true, three = true }
  -- default: false
  standalone_as_true = true,

  -- { key = { 'value' } } -> { key = 'value' }
  unpack_single_array_value = false,
}
local result = luakeys.parse('one,two,three', opts)
\end{minted}

\noindent
The options can also be set globally using the exported table
|default_options|:

\begin{minted}{lua}
luakeys.parse('dim=1cm') -- {dim = 1864679}
luakeys.default_options.convert_dimensions = false
-- or:
-- local defaults = luakeys.default_options
-- defaults.convert_dimensions = false
luakeys.parse('dim=1cm') -- {dim = '1cm'}
\end{minted}

%%
%
%%

\subsection{\texttt{define(defs, opts)}}

The \lua{define} function returns a \lua{parse} function (see
\ref{parse}).
The name of a key can be specified in three ways:

\begin{enumerate}
\item as standalone string values also known as an array of strings.
\item as keys in a Lua table. The definition of the corresponding
key-value pair is then stored under these keys.
\item by the “name” option.
\end{enumerate}

\begin{minted}{lua}
-- standalone string values
local defs = { 'key' }

-- keys in a Lua table
local defs = { key = { some_def = 'etc' } }

-- by the “name” option
local defs = { { name = 'key', some_def = 'etc' } }

local parse = luakeys.define(defs)
local result, leftover = parse('key=value,unknown=unknown')
luakeys.print(result) -- { key = 'value' }
luakeys.print(leftover) -- { unknown = 'unknown' }
\end{minted}

\noindent
For nested definitions, only the last two ways of specifying the key
names can be used.

\begin{minted}{lua}
local parse = luakeys.define({
  level1 = { sub_keys = { level2 = { sub_keys = { key = { some_def = 'etc' } } } } },
})
local result, leftover = parse('level1={level2={key=value,unknown=unknown}}')
luakeys.print(result) -- { level1 = { level2 = { key = 'value' } } })
luakeys.print(leftover) -- { level1 = { level2 = { unknown = 'unknown' } } })
\end{minted}

The code example below lists all the settings that can be used to define
key-value pairs.

\begin{minted}{lua}
local defs = {
  key = {
    -- Allow different key names.
    -- or a single string: alias = 'k'
    alias = { 'k', 'ke' },

    -- The key is always included in the result. If no default value is
    -- definied, true is taken as the value.
    always_present = false,

    -- Only values listed in the array table are allowed.
    choices = { 'one', 'two', 'three' },
    data_type = 'string', -- or boolean, integer,
    default = 'value',

    --
    exclusive_group = 'name',

    -- > \g_my_token_list_tl
    l3_tl_set = 'my_token_list',

    -- > \MacroName
    macro = 'MacroName', -- > \MacroName

    -- See http://www.lua.org/manual/5.3/manual.html#6.4.1
    match = '^%d%d%d%d%-%d%d%-%d%d$',

    -- name of the key, can be omitted
    name = 'key',
    opposite_values = { [true] = 'show', [false] = 'hide' },
    process = function(value, result, leftover)
      return value
    end,
    required = true,
    sub_keys = { key_level_2 = { ... } },
  }
}
\end{minted}

%%
%
%%

\subsubsection{\texttt{alias}}

With the help of the \lua{alias} option, other key names can be used.
The value is always stored under the original key name. A single alias
name can be specified by a string, multiple aliases by a list of
strings.

\begin{minted}{lua}
-- a single alias
local defs = { key = { alias = 'k' } }
-- multiple aliases
local defs = { key = { alias = { 'k', 'ke' } } }
local parse = luakeys.define(defs)
local result, leftover = parse('k=value')
luakeys.print(result) -- { key = 'value' }
\end{minted}

%%
%
%%

\subsubsection{\texttt{always\_present}}

The key is always included in the result. If no default value is
definied, true is taken as the value.

\begin{minted}{lua}
local parse = luakeys.define({ key = { always_present = true })
local result, leftover = parse('')
luakeys.print(result) -- { key =  true }
\end{minted}

\subsubsection{\texttt{opposite\_values}}

\begin{minted}{lua}
local keys = {
  visibility = {
    opposite_values = {
      [true] = 'show',
      [false] = 'hide'
    }
  }
}
\end{minted}

%%
%
%%

\subsubsection{\texttt{macro}}

The option \texttt{macro} stores the value in a \TeX macro.

\begin{minted}{lua}
local parse = luakeys.define({
  key = {
    macro = 'MyMacro'
  }
})
parse('key=value')
\end{minted}

\begin{minted}{latex}
\MyMacro % expands to “value”
\end{minted}

%%
%
%%

\subsubsection{\texttt{process}}

The “process” option can be used to define a function whose return value
is passed to the key. Three parameters are passed when the function is
called:

\begin{enumerate}
\item \lua{value}: The current value
\item \lua{result}: The table in which the final result will be saved.
% Da Lua keine feste Reihenfolge hat, wie die Schlüssel in einer Tabelle
% aufgerufen werden, kann diese Tabelle von Aufruf zu Aufruf jeweils
% ganz anderes aufgebaut sein.
Since Lua has no fixed order in which the keys in a table are called,
this table can be structured quite differently from call to call.
\item \lua{leftover}: The table in which the leftover key-value pairs
are stored.
\end{enumerate}

%%
%
%%

\begin{minted}{lua}
local defs = {
  key = {
    process = function (value, result, leftover)
      return value
    end
  }
}
\end{minted}

\subsubsection{\texttt{sub\_keys}}

\begin{minted}{lua}
local keys = {
  {
    name = 'key_level_1',
    sub_keys = {
      {
        name = 'key_level_2'
      }
    }
  }
}
\end{minted}

%%
%
%%

\subsection{\texttt{render(tbl)}: string}

The function \lua{render(tbl)} reverses the function
\lua{parse(kv_string)}. It takes a Lua table and converts this table
into a key-value string. The resulting string usually has a different
order as the input table.

\begin{minted}{lua}
result = luakeys.parse('one=1,two=2,tree=3,')
print(luakeys.render(result))
--- one=1,two=2,tree=3,
--- or:
--- two=2,one=1,tree=3,
--- or:
--- ...
\end{minted}

\noindent
In Lua only tables with 1-based consecutive integer keys (a.k.a. array
tables) can be parsed in order.

\begin{minted}{lua}
result = luakeys.parse('one,two,three')
print(luakeys.render(result))
--- one,two,three, (always)
\end{minted}

%%
%
%%

\subsection{\texttt{print(tbl): void}}

The function \lua{print(tbl)} pretty prints a Lua table to standard
output (stdout). It is a utility function that can be used to debug and
inspect the resulting Lua table of the function \lua{parse}. You have to
compile your \TeX{} document in a console to see the terminal output.

%\luakeysdebug{level1={level2={key=value}}}

\begin{minted}{lua}
result = luakeys.parse('level1={level2={key=value}}')
luakeys.print(result)
\end{minted}

\noindent
The output should look like this:

\begin{minted}{md}
{
  ['level1'] = {
    ['level2'] = {
      ['key'] = 'value',
  },
}
\end{minted}

%%
%
%%

\subsection{\texttt{save(identifier, result): void}}

The function \lua{save(identifier, result)} saves a result (a
table from a previous run of \lua{parse}) under an identifier.
Therefore, it is not necessary to pollute the global namespace to
store results for the later usage.

%%
%
%%

\subsection{\texttt{get(identifier): table}}

The function \lua{get(identifier)} retrieves a saved result from the
result store.

%-----------------------------------------------------------------------
%
%-----------------------------------------------------------------------

\section{Syntax of the recognized key-value format}

%%
%
%%

\subsection{An attempt to put the syntax into words}

A key-value pair is definied by an equal sign (\texttt{key=value}).
Several key-value pairs or values without keys are lined up with commas
(\texttt{key=value,value}) and build a key-value list. Curly brackets
can be used to create a recursive data structure of nested key-value
lists (\texttt{level1=\{level2=\{key=value,value\}\}}).

%%
%
%%

\subsection{An (incomplete) attempt to put the syntax into the Extended Backus-Naur Form}

\begin{grammar}
<list> ::= \{ <list-item> \}

<list-container> ::= `{' <list> `}'

<list-item> ::= ( <list-container> | <key-value-pair> | <value> ) [ `,' ]

<key-value-pair> ::= <value> `=' ( <list-container> | <value> )

<value> ::= <boolean>
  \alt <dimension>
  \alt <number>
  \alt <string-quoted>
  \alt <string-unquoted>

<sign> ::= `-' | `+'

<integer> ::= `0' | `1' | `2' | `3' | `4' | `5' | `6' | `7' | `8' | `9'

<unit> ::= `bp' | `BP'
  \alt `cc' | `CC'
  \alt `cm' | `CM'
  \alt `dd' | `DD'
  \alt `em' | `EM'
  \alt `ex' | `EX'
  \alt `in' | `IN'
  \alt `mm' | `MM'
  \alt `nc' | `NC'
  \alt `nd' | `ND'
  \alt `pc' | `PC'
  \alt `pt' | `PT'
  \alt `sp' | `SP'

<boolean> ::= <boolean-true> | <boolean-false>

<boolean-true> ::= `true' | `TRUE' | `True'

<boolean-false> ::= `false' | `FALSE' | `False'
\end{grammar}

... to be continued

%%
%
%%

\subsection{Recognized data types}

\subsubsection{boolean}

The strings \texttt{true}, \texttt{TRUE} and \texttt{True} are converted
into Lua’s boolean type \lua{true}, the strings \texttt{false},
\texttt{FALSE} and \texttt{False} into \lua{false}.

\begin{multicols}{2}
\begin{minted}{latex}
\luakeysdebug{
  lower case true = true,
  upper case true = TRUE,
  title case true = True
  lower case false = false,
  upper case false = FALSE,
  title case false = False,
}
\end{minted}
\begin{minted}{lua}
{
  ['lower case true'] = true,
  ['upper case true'] = true,
  ['title case true'] = true,
  ['lower case false'] = false,
  ['upper case false'] = false
  ['title case false'] = false,
}
\end{minted}
\end{multicols}

%%
%
%%

\clearpage

\subsubsection{number}

\begin{multicols}{2}
\begin{minted}{latex}
\luakeysdebug{
  num1 = 4,
  num2 = -4,
  num3 = 0.4
}
\end{minted}
\begin{minted}{lua}
{
  ['num1'] = 4,
  ['num2'] = -4,
  ['num3'] = 0.4
}
\end{minted}
\end{multicols}

%%
%
%%

\clearpage

\subsubsection{dimension}

|luakeys| detects \TeX{} dimensions and automatically converts the
dimensions into scaled points using the function \lua{tex.sp(dim)}. Use
the option \lua{convert_dimensions} of the function
\lua{parse(kv_string, options)} to disable the automatic conversion.

\begin{minted}{lua}
local result = parse('dim=1cm', {
  convert_dimensions = false,
})
\end{minted}

\noindent
If you want to convert a scale point into a unit string you can use the module
\href{https://raw.githubusercontent.com/latex3/lualibs/master/lualibs-util-dim.lua}{lualibs-util-dim.lua}.

\begin{minted}{latex}
\begin{luacode}
require('lualibs')
tex.print(number.todimen(tex.sp('1cm'), 'cm', '%0.0F%s'))
\end{luacode}
\end{minted}

\begin{center}
\begin{tabular}{rl}
\textbf{Unit name} & \textbf{Description} \\\hline
bp & big point \\
cc & cicero \\
cm & centimeter \\
dd & didot \\
em & horizontal measure of \emph{M} \\
ex & vertical measure of \emph{x} \\
in & inch \\
mm & milimeter \\
nc & new cicero \\
nd & new didot \\
pc & pica \\
pt & point \\
sp & scaledpoint \\
\end{tabular}
\end{center}

\begin{multicols}{2}
\begin{minted}{latex}
\luakeysdebug{
  bp = 1bp,
  cc = 1cc,
  cm = 1cm,
  dd = 1dd,
  em = 1em,
  ex = 1ex,
  in = 1in,
  mm = 1mm,
  nc = 1nc,
  nd = 1nd,
  pc = 1pc,
  pt = 1pt,
  sp = 1sp,
}
\end{minted}
\begin{minted}{lua}
{
  ['bp'] = 65781,
  ['cc'] = 841489,
  ['cm'] = 1864679,
  ['dd'] = 70124,
  ['em'] = 655360,
  ['ex'] = 282460,
  ['in'] = 4736286,
  ['mm'] = 186467,
  ['nc'] = 839105,
  ['nd'] = 69925,
  ['pc'] = 786432,
  ['pt'] = 65536,
  ['sp'] = 1,
}
\end{minted}
\end{multicols}

%%
%
%%

\subsubsection{string}

There are two ways to specify strings: With or without quotes. If the
text have to contain commas or equal signs, then double quotation
marks must be used.

\begin{multicols}{2}
\begin{minted}{latex}
\luakeysdebug{
  without quotes = no commas and equal signs are allowed,
  with double quotes = ", and = are allowed",
}
\end{minted}
\begin{minted}{lua}
{
  ['without quotes'] = 'no commas and equal signs are allowed',
  ['with double quotes'] = ', and = are allowed',
}
\end{minted}
\end{multicols}

\subsubsection{Standalone values}

Standalone values are values without a key. They are converted into an
array. In Lua an array is a table with numeric indexes (The first index
is 1).

\begin{multicols}{2}
\begin{minted}{latex}
\luakeysdebug{one,two,three}
\end{minted}
\columnbreak
\begin{minted}{lua}
{ 'one', 'two', 'three' }
\end{minted}

\noindent
is equivalent to

\begin{minted}{lua}
{
  [1] = 'one',
  [2] = 'two',
  [3] = 'three',
}
\end{minted}
\end{multicols}

\noindent
All recognized data types can be used as standalone values.

\begin{multicols}{2}
\begin{minted}{latex}
\luakeysdebug{one,2,3cm}
\end{minted}
\columnbreak
\begin{minted}{lua}
{ 'one', 2, 5594039 }
\end{minted}
\end{multicols}

%-----------------------------------------------------------------------
%
%-----------------------------------------------------------------------

\clearpage

\section{Process document class options}

\begin{minted}{latex}
\directlua{luakeys.parse('\@classoptionslist')}
\end{minted}

%-----------------------------------------------------------------------
%
%-----------------------------------------------------------------------

\clearpage

\section{Debug packages}

Two small debug packages are included in |luakeys|. One debug package
can be used in \LaTeX{} (luakeys-debug.sty) and one can be used in plain
\TeX{} (luakeys-debug.tex). Both packages provide only one command:
|\luakeysdebug{kv-string}|

\begin{minted}{latex}
\luakeysdebug{one,two,three}
\end{minted}

\noindent
Then the following output should appear in the document:
\bigskip

\luakeysdebug{one,two,three}

%%
%
%%

\subsection{For plain \TeX: luakeys-debug.tex}

An example of how to use the command in plain \TeX:

\begin{minted}{latex}
\input luakeys-debug.tex
\luakeysdebug{one,two,three}
\bye
\end{minted}

%%
%
%%

\subsection{For \LaTeX: luakeys-debug.sty}

An example of how to use the command in \LaTeX:

\begin{minted}{latex}
\documentclass{article}
\usepackage{luakeys-debug}
\begin{document}
\luakeysdebug[
  unpack single array values=false,
  convert dimensions=false
]{one,two,three}
\end{document}
\end{minted}

%-----------------------------------------------------------------------
%
%-----------------------------------------------------------------------

\clearpage

\section{Implementation}

%%
%
%%

\subsection{luakeys.lua}

\inputminted[linenos=true]{lua}{luakeys.lua}

%%
%
%%

\clearpage

\subsection{luakeys.tex}

\inputminted[linenos=true]{latex}{luakeys.tex}

%%
%
%%

\clearpage

\subsection{luakeys.sty}

\inputminted[linenos=true]{latex}{luakeys.sty}

%%
%
%%

\clearpage

\subsection{luakeys-debug.tex}

\inputminted[linenos=true]{latex}{luakeys-debug.tex}

%%
%
%%

\clearpage

\subsection{luakeys-debug.sty}

\inputminted[linenos=true]{latex}{luakeys-debug.sty}

\changes{v0.1}{2021/01/18}{Inital release}
\changes{v0.2}{2021/09/19}{
* Allow all recognized data types as keys
* Allow TeX macros in the values
* New public Lua functions: save(identifier, result), get(identifier)
}
\changes{v0.3}{2021/11/05}{
* Add a LuaLaTeX wrapper “luakeys.sty”
* Add a plain LuaTeX wrapper “luakeys.tex”
* Rename the previous documentation file “luakeys.tex” to luakeys-doc.tex”
}
\changes{v0.4}{2021/12/31}{
* Parser: Add support for nested tables (for example {{'a', 'b'}})
* Parser: Allow only strings and numbers as keys
* Parser: Remove support from Lua numbers with exponents (for example '5e+20')
* Switch the Lua testing framework to busted
}
\changes{v0.5}{2022/04/04}{
* Add possibility to change options globally
* New option: standalone\_as\_true
* Add a recursive converter callback / hook to process the parse tree
* New option: case\_insensitive\_keys
}
\pagebreak
\PrintChanges
\pagebreak
\PrintIndex
\end{document}
