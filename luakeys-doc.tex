\documentclass{ltxdoc}

\usepackage{hyperref}
\EnableCrossrefs
\CodelineIndex
\RecordChanges

\usepackage{mdframed}
\usepackage{minted}
\usepackage{luakeys-debug}
\usepackage{multicol}
\usepackage{luacode}
\usepackage{syntax}
\usemintedstyle{friendly}
\BeforeBeginEnvironment{minted}{\begin{mdframed}}
\AfterEndEnvironment{minted}{\end{mdframed}}
\setminted{
  breaklines=true,
  fontsize=\footnotesize,
  style=manni,
}
\def\lua#1{\mintinline{lua}|#1|}
\def\latex#1{\mintinline{latex}|#1|}

\NewDocumentCommand { \InputLatexExample } { O{} m } {
  \begin{mdframed}
  \inputminted[linenos=false,#1]{latex}{examples/#2}
  \end{mdframed}
}

\NewDocumentCommand { \InputLuaExample } { O{} m } {
  \begin{mdframed}
  \inputminted[linenos=false,#1]{lua}{examples/#2}
  \end{mdframed}
}

\begin{document}

\providecommand*{\url}{\texttt}

\title{The \textsf{luakeys} package}
\author{%
  Josef Friedrich\\%
  \url{josef@friedrich.rocks}\\%
  \href{https://github.com/Josef-Friedrich/luakeys}{github.com/Josef-Friedrich/luakeys}%
}
\date{v0.5 from 2022/04/04}

\maketitle

\vfill

\InputLuaExample[firstline=4,lastline=7]{first-page.lua}

\noindent
Result:

\begin{center}
\begin{minted}{lua}
{
  ['level1'] = {
    ['level2'] = {
      ['naked'] = true,
      ['dim'] = 1864679,
      ['bool'] = false,
      ['num'] = -0.001,
      ['str'] = 'lua,{}',
    }
  }
}
\end{minted}
\end{center}

\vfill

\strut

\newpage

\tableofcontents

\newpage

\section{Introduction}

\noindent
|luakeys| is a Lua module that can parse key-value options like the
\TeX{} packages \href{https://www.ctan.org/pkg/keyval}{keyval},
\href{https://www.ctan.org/pkg/kvsetkeys}{kvsetkeys},
\href{https://www.ctan.org/pkg/kvoptions}{kvoptions},
\href{https://www.ctan.org/pkg/xkeyval}{xkeyval},
\href{https://www.ctan.org/pkg/pgfkeys}{pgfkeys} etc. do. |luakeys|,
however, accomplishes this task entirely, by using the Lua language and
doesn’t rely on \TeX{}. Therefore this package can only be used with the
\TeX{} engine Lua\TeX{}. Since |luakeys| uses
\href{http://www.inf.puc-rio.br/~roberto/lpeg/}{LPeg}, the parsing
mechanism should be pretty robust.

The TUGboat article
\href{http://www.tug.org/tugboat/tb30-1/tb94wright-keyval.pdf}
{“Implementing key–value input: An introduction” (Volume 30 (2009), No. 1)}
by Joseph Wright and Christian Feuersänger gives a good overview of the
available key-value packages.

This package would not be possible without the article
\href{https://tug.org/TUGboat/tb40-2/tb125menke-lpeg.pdf}
{“Parsing complex data formats in LuaTEX with LPEG” (Volume 40 (2019), No. 2)}.

%-----------------------------------------------------------------------
%
%-----------------------------------------------------------------------

\section{How the package is loaded}

%%
%
%%

\subsection{Using the Lua module \texttt{luakeys.lua}}

The core functionality of this package is realized in Lua. So you can
use \texttt{luakeys} without using the wrapper \TeX{} files
\texttt{luakeys.sty} and \texttt{luakeys.tex}.

\InputLatexExample{loading/lua.tex}

%%
%
%%

\subsection{Using the Lua\LaTeX{} wrapper \texttt{luakeys.sty}}

The supplied Lua\LaTeX{} file is quite small:

\begin{minted}{latex}
\NeedsTeXFormat{LaTeX2e}
\ProvidesPackage{luakeys}
\directlua{luakeys = require('luakeys')}
\end{minted}

\noindent
It loads the Lua module into the global variable \texttt{luakeys}.

\InputLatexExample{loading/latex.tex}

%%
%
%%

\subsection{Using the plain Lua\TeX{} wrapper \texttt{luakeys.tex}}

Even smaller is the file \texttt{luakeys.tex}. It consists of only one
line:

\begin{minted}{latex}
\directlua{luakeys = require('luakeys')}
\end{minted}

\noindent
It does the same as the Lua\LaTeX{} wrapper and loads the Lua module
\texttt{luakeys.lua} into the global variable \texttt{luakeys}.

\InputLatexExample{loading/tex.tex}

\section{Lua interface / API}

To learn more about the individual functions (local functions), please
read the \href{https://josef-friedrich.github.io/luakeys/}{source code
documentation}, which was created with
\href{http://stevedonovan.github.io/ldoc/}{LDoc}. The Lua module exports
this functions and tables:

\InputLuaExample[firstline=3,lastline=12]{export.lua}

\subsection{Lua indentifier names}

% Das Projekt verwendet einige wenige Abkürzungen für Variablennamen,
% die hoffentlich für externe Leser eindeutig und bekannt sind.
The project uses a few abbreviations for variable names that are
hopefully unambiguous and familiar to external readers.

\begin{center}
\begin{tabular}{lll}
Abbreviation & spelled out & Example \\\hline
\lua{kv_string} & Key-value string & \lua{'key=value'} \\
\lua{opts} & Options (for the parse function) & \lua{ { no_error = false } } \\
\lua{defs} & Definitions \\
\lua{def} & Definition \\
\lua{attr} & Attributes (of a definition) \\
\end{tabular}
\end{center}

\noindent
% Diese nicht abgekürzten Variablennamen werden häufig verwendet.
These unabbreviated variable names are commonly used.
\begin{center}
\begin{tabular}{ll}

\lua{result} &
% Das Endergebnis aller einzelnen Übersetzungs- und Normalisierungsschritte
The final result of all individual parsing and normalization steps. \\

\lua{unknown} &
% Ein Tabelle mit unbekannten, nicht definierten Schlüssel-Wert-Paaren
A table with unknown, undefinied key-value pairs. \\

\lua{raw} &
% Das unbearbeitete, rohe Ergebnis der LPeg-Syntaxanalyse.
The raw result of the Lpeg grammar parser. \\
\end{tabular}
\end{center}

%%
%
%%

\subsection{Function \texttt{parse(kv\_string, opts): result, unknown, raw}}
\label{parse}

% Die Function parse ist die wichtigste Funktion des Pakets.
The function \lua{parse(kv_string, opts)} is the most important function
of the package.
% Sie konvertiert eine Schlüssel-Wert-Zeichenkette in eine Lua Tabelle.
It converts a key-value string into a Lua table.

\begin{minted}{latex}
\newcommand{\mykeyvalcmd}[2][]{
  \directlua{
    result = luakeys.parse('#1')
    luakeys.debug(result)
  }
  #2
}
\mykeyvalcmd[one=1]{test}
\end{minted}

\noindent
In plain \TeX:

\begin{minted}{latex}
\def\mykeyvalcommand#1{
  \directlua{
    result = luakeys.parse('#1')
    luakeys.debug(result)
  }
}
\mykeyvalcmd{one=1}
\end{minted}

\subsection{Options to configure the \texttt{parse} function}

\noindent
The \lua{parse} function can be called with an options table. This
options are supported:

\InputLuaExample[firstline=5,lastline=67]{opts/all-opts.lua}

\noindent
The options can also be set globally using the exported table
|opts|:

\InputLuaExample[firstline=4,lastline=4]{opts/exported-default-opts.lua}

\InputLuaExample[firstline=10,lastline=11]{opts/exported-default-opts.lua}

%%
%
%%

\subsubsection{Option “\texttt{convert\_dimensions}”}

If you set the option \lua{convert_dimensions} to \lua{true}, |luakeys|
detects the \TeX{} dimensions and converts them into scaled points using
the function \lua{tex.sp(dim)}.

\InputLuaExample[firstline=4,lastline=7]{opts/convert-dimensions.lua}

\noindent
By default the dimensions are not converted into scaled points.

\InputLuaExample[firstline=13,lastline=18]{opts/convert-dimensions.lua}

\noindent
If you want to convert a scale point into a unit string you can use the module
\href{https://raw.githubusercontent.com/latex3/lualibs/master/lualibs-util-dim.lua}{lualibs-util-dim.lua}.

\begin{minted}{lua}
require('lualibs')
tex.print(number.todimen(tex.sp('1cm'), 'cm', '%0.0F%s'))
\end{minted}

%%
%
%%

\subsubsection{Option “\texttt{debug}”}

If the option “debug” is set to ture, the result table is printed to the
console.

\begin{minted}{latex}
\documentclass{article}
\usepackage{luakeys}
\begin{document}
\directlua{
  luakeys.parse('one,two,three', { debug = true })
}
debug
\end{document}
\end{minted}

\begin{verbatim}
This is LuaHBTeX, Version 1.15.0 (TeX Live 2022)
...
(./debug.aux) (/usr/local/texlive/texmf-dist/tex/latex/base/ts1cmr.fd)
{
  ['three'] = true,
  ['two'] = true,
  ['one'] = true,
}
 [1{/usr/
local/texlive/2022/texmf-var/fonts/map/pdftex/updmap/pdftex.map}] (./debug.aux)
)
...
Transcript written on debug.log.
\end{verbatim}

%%
%
%%

\subsubsection{Option “\texttt{default}”}
\label{option-default}

% Mit der Option \lua{default} kann angegeben werden, welchen Wert
% nackte Schlüssel (Schlüssel ohne Wert) erhalten. Diese Option hat
% keinen Einfluss auf Schlüssel mit Werten.
The option \lua{default} can be used to specify which value naked keys
(keys without a value) get. This option has no influence on keys with
values.

\InputLuaExample[firstline=4,lastline=5]{opts/default.lua}

\noindent
% Standardmäßig erhalten nackte Schlüssel den Wert \lua{true}.
By default, naked keys get the value \lua{true}.

\InputLuaExample[firstline=11,lastline=12]{opts/default.lua}

%%
%
%%

\subsubsection{Option “\texttt{defaults}”}
\label{options-defaults}

% Mit der Attribut „defaults“ kann nicht nur ein einiger Standardwert
% angegeben werden, sondern eine ganze Tabelle mit Standardwerten.
The option “defaults” can be used to specify not only one default value,
but a whole table of default values.
% Die Ergebnistabelle wird mit der Tabelle bestehend aus Standardwerten
% vereinigt.
The result table is merged into the defaults table.
% Werte aus der Tabelle mit Standardwerten werden von Werten der
% Ergebnistabelle überschrieben.
Values in the defaults table are
overwritten by values in the result table.

\InputLuaExample[firstline=4,lastline=7]{opts/defaults.lua}

%%
%
%%

\subsubsection{Option “\texttt{defs}”}

% Für mehr Information wie Schlüssel definiert werden, lesen sie das
% kapitel 3.2
For more informations on how keys are defined, see section \ref{define}.
% Wenn sie die Option \lua{defs} verwenden, können sie auf den
% Aufruf der Funktion \lua{define} verzichten.
If you use the \lua{defs} option, you don't need to call the
\lua{define} function.
%
% Anstatt
Instead of ...

\InputLuaExample[firstline=4,lastline=5]{opts/defs.lua}

\noindent
% können wir schreiben ..
we can write ...

\InputLuaExample[firstline=11,lastline=13]{opts/defs.lua}

%%
%
%%

\subsubsection{Option “\texttt{format\_keys}”}

\begin{description}
\item[lower] \strut

\InputLuaExample[firstline=4,lastline=5]{opts/format-keys.lua}

\item[snake] \strut

\InputLuaExample[firstline=11,lastline=12]{opts/format-keys.lua}

\item[upper] \strut

\InputLuaExample[firstline=18,lastline=19]{opts/format-keys.lua}
\end{description}

%%
%
%%

\subsubsection{Option “\texttt{hooks}”}

% Die folgenden Hooks bzw. Callback-Funktionen ermöglichen es in den
% Verarbeitungsprozess der \lua{parse}-Function einzugreifen.
The following hooks or callback functions allow to intervene in the
processing of the \lua{parse} function.
%
% Die Funktionen sind in der Verarbeitungsreihenfolge aufgelistet.
The functions are listed in processing order.
%
% \lua{*_before_opts} bedeutet, dass die Hooks nach der LPeg Syntaxanalyse
% und vor dem Anwenden der Optionen ausgeführt
\lua{*_before_opts} means that the hooks are executed after the LPeg
syntax analysis and before the options are applied.
%
% Die Hooks \lua{*_before_defs} werden vor dem Anwenden der
% Schlüssel-Wert-Definitionen ausgeführt
The \lua{*_before_defs} hooks are executed before applying the key value
definitions.

\def\TmpSignature#1{
  {
    \scriptsize\texttt{ = #1}
  }
}

\def\TmpKeySignature{
  \TmpSignature{function(key, value, depth, current, result): key, value}
}
\def\TmpResultSignature{
  \TmpSignature{function(result): void}
}

\begin{enumerate}
\item \lua{kv_string} \TmpSignature{function(kv\_string): kv\_string}
\item \lua{keys_before_opts} \TmpKeySignature
\item \lua{result_before_opts} \TmpResultSignature
\item \lua{keys_before_def} \TmpKeySignature
\item \lua{result_before_def} \TmpResultSignature
\item (\lua{process}) (has to be definied using defs, see \ref{attr-process})
\item \lua{keys} \TmpKeySignature
\item \lua{result} \TmpResultSignature
\end{enumerate}

\paragraph{\texttt{kv\_string}}

Der Hook \lua{kv_string} wird als erste
der Hook-Funktionen noch vor der LPeg-Syntax-Analyse aufgerufen

The \lua{kv_string} hook is called as the first of the hook functions
before the LPeg syntax parser is executed.

\InputLuaExample[firstline=4,lastline=11]{hooks/kv-string.lua}

\paragraph{\texttt{keys\_*}}

% Die Hooks \lua{keys_*} werden rekursiv auf jeden Schlüssel in der
% aktuellen Ergebnistabelle aufgerufen.
The hooks \lua{keys_*} are called recursively on each key in the current
result table.
% Die Hook-Funktion muss zwei Werte zurückgeben: \lua{key, value}
The hook function must return two values: \lua{key, value}.
%
% Das folgende Beispiel gibt \lua{key} und \lua{value} unverändert zurück,
% sodass die Ergebnistabelle nicht verändert wird.
The following example returns \lua{key} and \lua{value} unchanged, so
the result table is not changed.

\InputLuaExample[firstline=4,lastline=11]{hooks/keys-unchanged.lua}

\noindent
% Das nächste Beispiel demonstriert den dritten Parameter \lua{depth}
% der Hook-Funktion.
The next example demonstrates the third parameter \lua{depth} of the
hook function.

\InputLuaExample[firstline=4,lastline=16]{hooks/keys-depth.lua}

\paragraph{\texttt{result\_*}}

% Die Hooks \lua{result_*} werden einmal mit der aktuellen
% Ergebnistabelle als Parameter aufgerufen.
The hooks \lua{result_*} are called once with the current result table
as a parameter.

%%
%
%%

\subsubsection{Option “\texttt{naked\_as\_value}”}

% Mit Hilfe der Option \lua{naked_as_value} werden nackte Schlüssel
% nicht mit einem Standardwert versehen, sondern als Werte in die
% Lua-Tabelle abgelegt.
With the help of the option \lua{naked_as_value}, naked keys are not
given a default value, but are stored as values in a Lua table.

\InputLuaExample[firstline=4,lastline=5]{opts/naked-as-value.lua}

\noindent
If we set the option \lua{naked_as_value} to \lua{true}:

\InputLuaExample[firstline=11,lastline=14]{opts/naked-as-value.lua}

%%
%
%%

\subsubsection{Option “\texttt{no\_error}”}

% Standardmaßig wirft parse-Funktion einen Fehler, wenn es unbekannte
% Schlüssel gibt.
By default the parse function throws an error if there are unknown keys.
% Mit Hilfe der Option \lua{no_error} kann dies unterbunden werden.
This can be prevented with the help of the \lua{no_error} option.

\InputLuaExample[firstline=5,lastline=6]{opts/no-error.lua}

\noindent
If we set the option \lua{no_error} to \lua{true}:

\InputLuaExample[firstline=9,lastline=10]{opts/no-error.lua}

%%
%
%%

\subsubsection{Option “\texttt{unpack}”}

% Mit Hilfe der Option \lua{unpack} werden alle Tabellen, die nur aus
% einem einzigen nackten Schlüssel bzw. einen einzigen alleinstehenden
% Wert bestehen, aufgelöst.
With the help of the option \lua{unpack}, all tables that consist of
only one a single naked key or a single standalone value are unpacked.

\InputLuaExample[firstline=4,lastline=5]{opts/unpack.lua}

\InputLuaExample[firstline=11,lastline=12]{opts/unpack.lua}

%%
%
%%

\subsection{Function \texttt{define(defs, opts): parse}}
\label{define}

The \lua{define} function returns a \lua{parse} function (see
\ref{parse}).
The name of a key can be specified in three ways:

\begin{enumerate}
\item as a string.
\item as a key in a Lua table. The definition of the corresponding
key-value pair is then stored under this key.
\item by the “name” attribute.
\end{enumerate}

\InputLuaExample[firstline=4,lastline=16]{functions/define.lua}

\noindent
For nested definitions, only the last two ways of specifying the key
names can be used.

\InputLuaExample[firstline=26,lastline=33]{functions/define.lua}

%-----------------------------------------------------------------------
%
%-----------------------------------------------------------------------

\subsection{Attributes to define a key-value pair}

% Die Definition eines Schlüssel-Wert-Paares kann mit Hilfe von
% verschiedenen Attributen vorgenommen werden.
The definition of a key-value pair can be made with the help of various
attributes.
%
% Der Name „Attribut“ für eine Option, einen Schlüssel, eine Eigenschaft
% (um nur einige Benennungsmöglichkeiten aufzuzählen) zur
% Schlüssel-Definition, wurde bewusst gewählt, um sie von den Optionen
% der Funktion \lua{parse} zu unterscheiden.
The name \emph{“attribute”} for an option, a key, a property ... (to
list just a few naming possibilities) to define keys, was deliberately
chosen to distinguish them from the options of the \lua{parse} function.
%
% Das folgende Codebeispiel listet alle Attribute auf, die verwendet
% werden können, um Schlüssel-Wert-Paare zu definieren.
The code example below lists all the attributes that can be used to
define key-value pairs.

\InputLuaExample[firstline=5,lastline=41]{defs/all-attrs.lua}

%%
%
%%

\subsubsection{Attribute “\texttt{alias}”}

With the help of the \lua{alias} attribute, other key names can be used.
The value is always stored under the original key name. A single alias
name can be specified by a string ...

\InputLuaExample[firstline=4,lastline=7]{defs/attrs/alias.lua}

\noindent
multiple aliases by a list of strings.

\InputLuaExample[firstline=13,lastline=16]{defs/attrs/alias.lua}

%%
%
%%

\subsubsection{Attribute “\texttt{always\_present}”}

% Die Option \lua{default} wird nur bei nackten Schlüsseln verwendet.
The \lua{default} attribute is used only for naked keys.

\InputLuaExample[firstline=4,lastline=5]{defs/attrs/always-present.lua}

\noindent
% Wird die Option \lua{always_present} auf wahr gesetzt, wird der
% Schlüssel immer ins Ergebnis mit übernommen.
If the attribute \lua{always_present} is set to true, the key is always
included in the result. If no default value is definied, true is taken
as the value.

\InputLuaExample[firstline=11,lastline=12]{defs/attrs/always-present.lua}

%%
%
%%

\subsubsection{Attribute “\texttt{choices}”}

% source: Python argparse documentation.
Some key values should be selected from a restricted set of choices.
These can be handled by passing an array table containing choices.

\InputLuaExample[firstline=4,lastline=5]{defs/attrs/choices.lua}

\noindent
When the key-value pair is parsed, values will be checked, and an error
message will be displayed if the value was not one of the acceptable
choices:

\InputLuaExample[firstline=13,lastline=15]{defs/attrs/choices.lua}

%%
%
%%

\subsubsection{Attribute “\texttt{data\_type}”}

% source: Python argparse documentation.
The \lua{data_type} attribute allows type-checking and type conversions to
be performed.
%
% Folgende Datentypen werden unterstützt:
The following data types are supported:
\lua{'boolean'},
\lua{'dimension'},
\lua{'integer'},
\lua{'number'},
\lua{'string'}.
%
% Bei den drei Datentypen integer, number, dimension kann eine
% Typenumwandlung scheitern.
A type conversion can fail with the three data types
\lua{'dimension'},
\lua{'integer'},
\lua{'number'}.
%
% Dann wird eine Fehlermeldung ausgegeben.
Then an error message is displayed.

\InputLuaExample[firstline=4,lastline=8]{defs/attrs/data-type.lua}
\InputLuaExample[firstline=11,lastline=15]{defs/attrs/data-type.lua}

%%
%
%%

\subsubsection{Attribute “\texttt{default}”}

% Verwenden Sie das Attribut „\lua{default}“, um für jeden nackten Schlüssel
% einzeln einen Standardwert bereit zu stellen.
Use the \lua{default} attribute to provide a default value for each naked
key individually.
%
% Mit der globalen \lua{default} Option kann für alle nackten Schlüssel ein
% Standardwert vorgegeben werden.
With the global \lua{default} attribute (\ref{option-default}) a default
value can be specified for all naked keys.

\InputLuaExample[firstline=4,lastline=9]{defs/attrs/default.lua}

%%
%
%%

\subsubsection{Attribute “\texttt{exclusive\_group}”}

% Alle Schlüssel, die der gleichen ausschließenden Gruppe angehören,
% dürfen nicht gemeinsam angegeben werden.
All keys belonging to the same exclusive group must not be specified
together.
%
% Nur ein Schlüssel aus dieser Gruppe ist erlaubt.
Only one key from this group is allowed.
%
% Als Name für diese ausschließende Gruppe kann irgend ein beliebiger
% Wert verwendet werden.
Any value can be used as a name for this exclusive group.

\InputLuaExample[firstline=4,lastline=9]{defs/attrs/exclusive-group.lua}

% Werden mehrer Schlüssel der Gruppe angegeben, so wird eine
% Fehlermeldung geworfen.
If more than one key of the group is specified, an error message is
thrown.

\InputLuaExample[firstline=21,lastline=23]{defs/attrs/exclusive-group.lua}

%%
%
%%

\subsubsection{Attribute “\texttt{opposite\_keys}”}

% Die Option \lua{opposite_keys} ermöglicht es, gegensätzliche (nackte)
% Schlüssel in Wahrheitswerte umzuwandeln und diesen Wahrheitswert unter
% einem Zielschlüssel zu speichern.
The \lua{opposite_keys} attribute allows to convert opposite (naked) keys
into a boolean value and store this boolean under a target key.
%
% Lua erlaubt es in Tabellen Wahrheitswerte als Schlüssel zu verwenden.
% Es müssen jedoch eckige Klammern verwendet werden.
Lua allows boolean values to be used as keys in tables.
%
% Die Wahrheitswerte müssen jedoch in eckige Klammern geschrieben werden.
However, the boolean values must be written in square brackets, e. g.
\lua{{ opposite_keys = { [true] = 'show', [false] = 'hide' } }}.
%
% Beispiele für gegensätzliche Schlüssel sind:
Examples of opposing keys are: \lua{show} and \lua{hide}, \lua{dark} and
\lua{light}, \lua{question} and \lua{solution}.
%
% Das untenstehende Beispiel verwendet als gegensätzliches Schlüsselpaar
% die Schlüssel \lua{show} und \lua{hide}.
The example below uses the \lua{show} and \lua{hide} keys as the
opposite key pair.
%
% Wird der Schlüssel \lua{show} von der Funktion \lua{parse} gelesen,
% dann erhält der Zielschlüssel \lua{visibility} den Wert \lua{true}.
If the key \lua{show} is parsed by the \lua{parse} function, then the
target key \lua{visibility} receives the value \lua{true}.

\InputLuaExample[firstline=4,lastline=7]{defs/attrs/opposite-keys.lua}

% Wird der Schlüssel \lua{hide} gelesen, dann \lua{falsch}.
\noindent
If the key \lua{hide} is parsed, then \lua{false}.

\InputLuaExample[firstline=13,lastline=13]{defs/attrs/opposite-keys.lua}

%%
%
%%

\subsubsection{Attribute “\texttt{macro}”}

The attribute \texttt{macro} stores the value in a \TeX{} macro.

\begin{minted}{lua}
local parse = luakeys.define({
  key = {
    macro = 'MyMacro'
  }
})
parse('key=value')
\end{minted}

\begin{minted}{latex}
\MyMacro % expands to “value”
\end{minted}

%%
%
%%

\subsubsection{Attribute “\texttt{match}”}

% Der Wert des Schlüssel wird der Lua Funktion übergeben
The value of the key is passed to the Lua function
\lua{string.match(value, match)} ((\url{http://www.lua.org/manual/5.3/manual.html#pdf-string.match})).
% Werfe einen Blick in das Lua-Handbuch, wie man Patterns schreibt.
Take a look at the Lua manual on how to write patterns
(\url{http://www.lua.org/manual/5.3/manual.html#6.4.1})

\InputLuaExample[firstline=4,lastline=6]{defs/attrs/match.lua}

\noindent
% Kann das Pattern im Wert nicht gefunden werden, wird eine
% Fehlermeldung ausgegeben.
If the pattern cannot be found in the value, an error message is issued.

\InputLuaExample[firstline=14,lastline=17]{defs/attrs/match.lua}

\noindent
% Der Schlüssel erhält das Ergebnis der Funktion \lua{string.match(value,
% match)}, dass bedeutet, dass der ursprüngliche Wert unter
% Umständen nicht vollständig in den Schlüssel gespeichert wird.
The key receives the result of the function \lua{string.match(value,
match)}, which means that the original value may not be stored
completely in the key.

\InputLuaExample[firstline=22,lastline=23]{defs/attrs/match.lua}

\noindent
% Das Präfix “waste ” und das Suffix “ rubbisch” der Zeichenketten wird
% verworfen.
The prefix “waste ” and the suffix “ rubbisch” of the string are
discarded.

\InputLuaExample[firstline=29,lastline=29]{defs/attrs/match.lua}

\noindent
% Da Funktion \lua{string.match(value, match)} immer eine Zeichenkette
% zurückgibt, ist der Wert des Schlüssel auch immer eine Zeichenkette.
Since function \lua{string.match(value, match)} always returns a string,
the value of the key is also always a string.

%%
%
%%

\subsubsection{Attribute “\texttt{name}”}

% Die Option \lua{name} ermöglicht eine alternative Notation von
% Schlüsselnamen.
The \lua{name} attribute allows an alternative notation of key names.
%
% Anstatt ...
Instead of ...

\InputLuaExample[firstline=4,lastline=5]{defs/attrs/name.lua}

\noindent
% ... können wir schreiben:
... we can write:

\InputLuaExample[firstline=11,lastline=15]{defs/attrs/name.lua}

%%
%
%%

\subsubsection{Attribute “\texttt{process}”}
\label{attr-process}

The \lua{process} attribute can be used to define a function whose return
value is passed to the key. Four parameters are passed when the
function is called:

\begin{enumerate}
\item \lua{value}:
% Der zum schlüssel gehörende aktuelle Wert.
The current value asssociated with the key.

\item \lua{input}:
% Die Ergebnis-Tabelle, die vor dem Zeitpunkt geklont wurde, als mit dem
% Anwenden der Definitionen begonnen wurde.
The result table cloned before the time the definitions started to be applied.

\item \lua{result}: The table in which the final result will be saved.

\item \lua{unknown}: The table in which the unknown key-value pairs
are stored.
\end{enumerate}

% Das folgende Beispiel demonstriert den Parameter \lua{value}.
\noindent
The following example demonstrates the \lua{value} parameter:
\InputLuaExample[firstline=4,lastline=14]{defs/attrs/process.lua}

\noindent
The following example demonstrates the \lua{input} parameter:

\InputLuaExample[firstline=22,lastline=34]{defs/attrs/process.lua}

\noindent
The following example demonstrates the \lua{result} parameter:

\InputLuaExample[firstline=42,lastline=50]{defs/attrs/process.lua}

\noindent
The following example demonstrates the \lua{unknown} parameter:

\InputLuaExample[firstline=58,lastline=65]{defs/attrs/process.lua}
\InputLuaExample[firstline=69,lastline=69]{defs/attrs/process.lua}

%%
%
%%

\subsubsection{Attribute “\texttt{required}”}

\InputLuaExample[firstline=4,lastline=5]{defs/attrs/required.lua}

\InputLuaExample[firstline=13,lastline=14]{defs/attrs/required.lua}

\noindent
A recursive example:

\InputLuaExample[firstline=18,lastline=23]{defs/attrs/required.lua}
\InputLuaExample[firstline=29,lastline=30]{defs/attrs/required.lua}
\InputLuaExample[firstline=38,lastline=39]{defs/attrs/required.lua}

%%
%
%%

\subsubsection{Attribute “\texttt{sub\_keys}”}

% Mit dem Attribut \lua{sub_keys} können ineinander verschachtelte
% Schlüssel-Wert-Paar-Definitionen aufgebaut werden.
The \lua{sub_keys} attribute can be used to build nested key-value pair
definitions.

\InputLuaExample[firstline=4,lastline=16]{defs/attrs/sub-keys.lua}

%-----------------------------------------------------------------------
%
%-----------------------------------------------------------------------

\subsection{Function \texttt{render(result): string}}

The function \lua{render(result)} reverses the function
\lua{parse(kv_string)}. It takes a Lua table and converts this table
into a key-value string. The resulting string usually has a different
order as the input table.

\InputLuaExample[firstline=4,lastline=10]{functions/render.lua}

\noindent
In Lua only tables with 1-based consecutive integer keys (a.k.a. array
tables) can be parsed in order.

\InputLuaExample[firstline=16,lastline=17]{functions/render.lua}

%%
%
%%

\subsection{Function \texttt{debug(result): void}}

The function \lua{debug(result)} pretty prints a Lua table to standard
output (stdout). It is a utility function that can be used to debug and
inspect the resulting Lua table of the function \lua{parse}. You have to
compile your \TeX{} document in a console to see the terminal output.

\InputLuaExample[firstline=4,lastline=5]{functions/debug.lua}

\noindent
The output should look like this:

\begin{minted}{md}
{
  ['level1'] = {
      ['level2'] = {
        ['key'] = 'value',
    },
  }
}
\end{minted}

%%
%
%%

\subsection{Function \texttt{save(identifier, result): void}}

The function \lua{save(identifier, result)} saves a result (a
table from a previous run of \lua{parse}) under an identifier.
Therefore, it is not necessary to pollute the global namespace to
store results for the later usage.

%%
%
%%

\subsection{Function \texttt{get(identifier): result}}

The function \lua{get(identifier)} retrieves a saved result from the
result store.

%%
%
%%

\subsection{Table \texttt{is}}

\subsubsection{Function \texttt{is.boolean(value): boolean}}
\InputLuaExample[firstline=7,lastline=23]{is-table.lua}

\subsubsection{Function \texttt{is.dimension(value): boolean}}
\InputLuaExample[firstline=27,lastline=37]{is-table.lua}

\subsubsection{Function \texttt{is.integer(value): boolean}}
\InputLuaExample[firstline=41,lastline=46]{is-table.lua}

\subsubsection{Function \texttt{is.number(value): boolean}}
\InputLuaExample[firstline=50,lastline=57]{is-table.lua}

\subsubsection{Function \texttt{is.string(value): boolean}}
\InputLuaExample[firstline=61,lastline=67]{is-table.lua}

%-----------------------------------------------------------------------
%
%-----------------------------------------------------------------------

\section{Syntax of the recognized key-value format}

%%
%
%%

\subsection{An attempt to put the syntax into words}

A key-value pair is definied by an equal sign (\texttt{key=value}).
Several key-value pairs or keys without values (naked keys) are lined up
with commas (\texttt{key=value,naked}) and build a key-value list. Curly
brackets can be used to create a recursive data structure of nested
key-value lists (\texttt{level1=\{level2=\{key=value,naked\}\}}).

%%
%
%%

\subsection{An (incomplete) attempt to put the syntax into the Extended Backus-Naur Form}

\begin{grammar}
<list> ::= \{ <list-item> \}

<list-container> ::= `{' <list> `}'

<list-item> ::= ( <list-container> | <key-value-pair> | <value> ) [ `,' ]

<key-value-pair> ::= <value> `=' ( <list-container> | <value> )

<value> ::= <boolean>
  \alt <dimension>
  \alt <number>
  \alt <string-quoted>
  \alt <string-unquoted>

<dimension> ::= <number> <unit>

<number> ::= <sign> ( <integer> [ <fractional> ] | <fractional> )

<fractional> ::= `.' <integer>

<sign> ::= `-' | `+'

<integer> ::= <digit> \{ <digit> \}

<digit> ::= `0' | `1' | `2' | `3' | `4' | `5' | `6' | `7' | `8' | `9'

<unit> ::= `bp' | `BP'
  \alt `cc' | `CC'
  \alt `cm' | `CM'
  \alt `dd' | `DD'
  \alt `em' | `EM'
  \alt `ex' | `EX'
  \alt `in' | `IN'
  \alt `mm' | `MM'
  \alt `nc' | `NC'
  \alt `nd' | `ND'
  \alt `pc' | `PC'
  \alt `pt' | `PT'
  \alt `sp' | `SP'

<boolean> ::= <boolean-true> | <boolean-false>

<boolean-true> ::= `true' | `TRUE' | `True'

<boolean-false> ::= `false' | `FALSE' | `False'
\end{grammar}

... to be continued

%%
%
%%

\subsection{Recognized data types}

\subsubsection{boolean}

The strings \texttt{true}, \texttt{TRUE} and \texttt{True} are converted
into Lua’s boolean type \lua{true}, the strings \texttt{false},
\texttt{FALSE} and \texttt{False} into \lua{false}.

\begin{multicols}{2}
\begin{minted}{latex}
\luakeysdebug{
  lower case true = true,
  upper case true = TRUE,
  title case true = True
  lower case false = false,
  upper case false = FALSE,
  title case false = False,
}
\end{minted}
\begin{minted}{lua}
{
  ['lower case true'] = true,
  ['upper case true'] = true,
  ['title case true'] = true,
  ['lower case false'] = false,
  ['upper case false'] = false
  ['title case false'] = false,
}
\end{minted}
\end{multicols}

%%
%
%%

\subsubsection{number}

\begin{multicols}{2}
\begin{minted}{latex}
\luakeysdebug{
  num1 = 4,
  num2 = -4,
  num3 = 0.4
}
\end{minted}
\begin{minted}{lua}
{
  ['num1'] = 4,
  ['num2'] = -4,
  ['num3'] = 0.4
}
\end{minted}
\end{multicols}

%%
%
%%

\subsubsection{dimension}

\begin{center}
\begin{tabular}{rl}
\textbf{Unit name} & \textbf{Description} \\\hline
bp & big point \\
cc & cicero \\
cm & centimeter \\
dd & didot \\
em & horizontal measure of \emph{M} \\
ex & vertical measure of \emph{x} \\
in & inch \\
mm & milimeter \\
nc & new cicero \\
nd & new didot \\
pc & pica \\
pt & point \\
sp & scaledpoint \\
\end{tabular}
\end{center}

\begin{multicols}{2}
\begin{minted}{latex}
\luakeysdebug{
  bp = 1bp,
  cc = 1cc,
  cm = 1cm,
  dd = 1dd,
  em = 1em,
  ex = 1ex,
  in = 1in,
  mm = 1mm,
  nc = 1nc,
  nd = 1nd,
  pc = 1pc,
  pt = 1pt,
  sp = 1sp,
}
\end{minted}
\begin{minted}{lua}
{
  ['bp'] = 65781,
  ['cc'] = 841489,
  ['cm'] = 1864679,
  ['dd'] = 70124,
  ['em'] = 655360,
  ['ex'] = 282460,
  ['in'] = 4736286,
  ['mm'] = 186467,
  ['nc'] = 839105,
  ['nd'] = 69925,
  ['pc'] = 786432,
  ['pt'] = 65536,
  ['sp'] = 1,
}
\end{minted}
\end{multicols}

%%
%
%%

\subsubsection{string}

There are two ways to specify strings: With or without quotes. If the
text have to contain commas or equal signs, then double quotation
marks must be used.

\begin{multicols}{2}
\begin{minted}{latex}
\luakeysdebug{
  without quotes = no commas and equal signs are allowed,
  with double quotes = ", and = are allowed",
}
\end{minted}
\begin{minted}{lua}
{
  ['without quotes'] = 'no commas and equal signs are allowed',
  ['with double quotes'] = ', and = are allowed',
}
\end{minted}
\end{multicols}

\subsubsection{Naked keys}

% Nackte Schlüssel sind Schlüssel ohne Wert.
Naked keys are keys without a value.
%
% Mit der Option \lua{naked_as_value} können sie als Werte in ein Feld
% übernommen werden.
Using the option \lua{naked_as_value} they can be converted into values
and stored into an array.
%
% In Lua ist ein Feld eine Tabelle mit numerischen Indizes (der erste
% Index ist 1).
In Lua an array is a table with numeric indexes (The first index is 1).

\InputLatexExample[firstline=5,lastline=12]{luakeysdebug/naked-keys.tex}

\noindent
% VAlle erkannten Datentypen können als eigenständige Werte verwendet
% werden.
All recognized data types can be used as standalone values.

\InputLatexExample[firstline=14,lastline=19]{luakeysdebug/naked-keys.tex}

%-----------------------------------------------------------------------
%
%-----------------------------------------------------------------------

\clearpage

\section{Examples}

\subsection{Extend and modify keys of existing macros}

Extend the includegraphics macro with a new key named \latex{caption}
and change the accepted values of the \latex{width} key. A number
between 0 and 1 is allowed and converted into
\latex{width=0.5\linewidth}

\InputLuaExample{extend-includegraphics/extend-keys.lua}
\InputLatexExample{extend-includegraphics/extend-keys.tex}

\subsection{Process document class options}

\begin{minted}{latex}
\directlua{luakeys.parse('\@classoptionslist')}
\end{minted}

\InputLatexExample{class-options/test-class.cls}
\InputLatexExample{class-options/use-test-class.tex}

\begin{minted}{lua}
{
  [1] = '12pt',
  [2] = 'landscape',
}
\end{minted}

%-----------------------------------------------------------------------
%
%-----------------------------------------------------------------------

\clearpage

\section{Debug packages}

Two small debug packages are included in |luakeys|. One debug package
can be used in \LaTeX{} (luakeys-debug.sty) and one can be used in plain
\TeX{} (luakeys-debug.tex). Both packages provide only one command:
|\luakeysdebug{kv-string}|

\begin{minted}{latex}
\luakeysdebug{one,two,three}
\end{minted}

\noindent
Then the following output should appear in the document:
\bigskip

\luakeysdebug{one,two,three}

%%
%
%%

\subsection{For plain \TeX: luakeys-debug.tex}

An example of how to use the command in plain \TeX:

\begin{minted}{latex}
\input luakeys-debug.tex
\luakeysdebug{one,two,three}
\bye
\end{minted}

%%
%
%%

\subsection{For \LaTeX: luakeys-debug.sty}

An example of how to use the command in \LaTeX:

\begin{minted}{latex}
\documentclass{article}
\usepackage{luakeys-debug}
\begin{document}
\luakeysdebug[
  unpack=false,
  convert dimensions=false
]{one,two,three}
\end{document}
\end{minted}

%-----------------------------------------------------------------------
%
%-----------------------------------------------------------------------

\clearpage

\section{Implementation}

%%
%
%%

\subsection{luakeys.lua}

\inputminted[linenos=true]{lua}{luakeys.lua}

%%
%
%%

\clearpage

\subsection{luakeys.tex}

\inputminted[linenos=true]{latex}{luakeys.tex}

%%
%
%%

\clearpage

\subsection{luakeys.sty}

\inputminted[linenos=true]{latex}{luakeys.sty}

%%
%
%%

\clearpage

\subsection{luakeys-debug.tex}

\inputminted[linenos=true]{latex}{luakeys-debug.tex}

%%
%
%%

\clearpage

\subsection{luakeys-debug.sty}

\inputminted[linenos=true]{latex}{luakeys-debug.sty}

\changes{v0.1}{2021/01/18}{Inital release}
\changes{v0.2}{2021/09/19}{
* Allow all recognized data types as keys
* Allow TeX macros in the values
* New public Lua functions: save(identifier, result), get(identifier)
}
\changes{v0.3}{2021/11/05}{
* Add a LuaLaTeX wrapper “luakeys.sty”
* Add a plain LuaTeX wrapper “luakeys.tex”
* Rename the previous documentation file “luakeys.tex” to luakeys-doc.tex”
}
\changes{v0.4}{2021/12/31}{
* Parser: Add support for nested tables (for example {{'a', 'b'}})
* Parser: Allow only strings and numbers as keys
* Parser: Remove support from Lua numbers with exponents (for example '5e+20')
* Switch the Lua testing framework to busted
}
\changes{v0.5}{2022/04/04}{
* Add possibility to change options globally
* New option: standalone\_as\_true
* Add a recursive converter callback / hook to process the parse tree
* New option: case\_insensitive\_keys
}
\changes{v0.6}{2022/05/29}{
* Rename global options table from “default\_options” to
* New function “define(defs, opts)”
* New option “format\_keys”
* Remove option “case\_insensitive\_keys”. Use
  “format_keys = \{ lower \}” to achieve the same effect.
* The default value of the option “convert\_dimension” is now false.
* The function “print” is now called “debug”.
* The option “standalone\_as\_true” is renamed to “naked\_as\_value”. The
  boolean value of the option must be changed to the opposite to
  produce the previous effect.
}
\pagebreak
\PrintChanges
\pagebreak
\PrintIndex
\end{document}
