\documentclass{ltxdoc}

\usepackage[
  colorlinks=true,
  linkcolor=blue,
  filecolor=blue,
  urlcolor=blue,
]{hyperref}
\EnableCrossrefs
\CodelineIndex
\RecordChanges

\usepackage{mdframed}
\usepackage{minted}
\usepackage{luakeys-debug}
\usepackage{multicol}
\usemintedstyle{friendly}
\BeforeBeginEnvironment{minted}{\begin{mdframed}}
\AfterEndEnvironment{minted}{\end{mdframed}}
\setminted{
  breaklines=true,
  fontsize=\footnotesize,
}
\def\lua#1{\mintinline{lua}|#1|}

\begin{document}

\providecommand*{\url}{\texttt}

\title{The \textsf{luakeys} package}
\author{%
  Josef Friedrich\\%
  \url{josef@friedrich.rocks}\\%
  \href{https://github.com/Josef-Friedrich/nodetree}{github.com/Josef-Friedrich/luakeys}%
}
\date{v0.1 from 2021/01/10}

\maketitle

\vfill

\luakeysdebug{level1={level2={level3={dim=1cm,bool=true,num=-1e-03,str=lua}}}}

\begin{minted}{lua}
local luakeys = require('luakeys')
local kv = luakeys.parse('level1={level2={level3={dim=1cm,bool=true,num=-1e-03,str=lua}}}')
luakeys.print(kv)
\end{minted}

\noindent
Result:

\begin{center}
\begin{minted}{lua}
{
  ['level1'] = {
    ['level2'] = {
      ['level3'] = {
        ['dim'] = 1864679,
        ['bool'] = true,
        ['num'] = -0.001
        ['str'] = 'lua',
      }
    }
  }
}
\end{minted}
\end{center}

\vfill

\strut

\newpage

\tableofcontents

\newpage

\section{Introduction}

\noindent
|luakeys| is a Lua module that can parse key-value options like the
\TeX{} packages \href{https://www.ctan.org/pkg/keyval}{keyval},
\href{https://www.ctan.org/pkg/kvsetkeys}{kvsetkeys},
\href{https://www.ctan.org/pkg/kvoptions}{kvoptions},
\href{https://www.ctan.org/pkg/xkeyval}{xkeyval},
\href{https://www.ctan.org/pkg/pgfkeys}{pgfkeys} etc. But |luakeys|
accompilshes this task entirely using the Lua language and doesn’t rely
on \TeX{} macros. Therefore this package can only be used with the
\TeX{} engine Lua\TeX{}. Since |luakeys| uses
\href{http://www.inf.puc-rio.br/~roberto/lpeg/}{LPeg}, the parsing
mechanism should be pretty robust.

\href{http://www.tug.org/tugboat/tb30-1/tb94wright-keyval.pdf}{“Implementing key–value input: An introduction” (TUGboat, Volume 30 (2009), No. 1)}
by Joseph Wright and Christian Feuersänger

%-----------------------------------------------------------------------
%
%-----------------------------------------------------------------------

\section{Recognized data types}

%%
%
%%

\subsection{boolean}

\begin{multicols}{2}
\begin{minted}{latex}
\luakeysdebug{
  lower case true = true,
  upper case true = TRUE,
  title case true = True
  lower case false = false,
  upper case false = FALSE,
  title case false = False,
}
\end{minted}
\begin{minted}{lua}
{
  ['lower case true'] = true,
  ['upper case true'] = true,
  ['title case true'] = true,
  ['lower case false'] = false,
  ['upper case false'] = false
  ['title case false'] = false,
}
\end{minted}
\end{multicols}

%%
%
%%

\subsection{number}

\begin{multicols}{2}
\begin{minted}{latex}
\luakeysdebug{
  num1 = 4,
  num2 = 4,
  num3 = 0.4,
  num4 = 4.57e-3,
  num5 = 0.3e12,
  num6 = 5e+20
}
\end{minted}
\begin{minted}{lua}
{
  ['num1'] = 4,
  ['num2'] = 4,
  ['num3'] = 0.4,
  ['num4'] = 0.00457,
  ['num5'] = 300000000000.0,
  ['num6'] = 5e+20
}
\end{minted}
\end{multicols}

%%
%
%%

\subsection{dimension}

|luakeys| detects \TeX{} dimensions and automatically converts the
dimensions into scaled points using the function \lua{tex.sp(dim)}. Use
the option \lua{convert_dimensions} of the function
\lua{parse(kv_string, options)} to disalbe the automatic conversion.

\begin{minted}{lua}
local result = parse('dim=1cm', {
  convert_dimensions = false,
})
\end{minted}

If you want to convert a scale point into a unit string you can used the module
\href{https://raw.githubusercontent.com/latex3/lualibs/master/lualibs-util-dim.lua}{lualibs-util-dim.lua}.

\directlua{
local dim = require('lualibs')
print('-----------------------------------------------')
print(dim)
print(number.tocentimeters(1864679))
tex.print(number.tocentimeters(1864679, 'lol'))
}

function number.topoints      (n,fmt) return numbertodimen(n,"pt",fmt) end
function number.toinches      (n,fmt) return numbertodimen(n,"in",fmt) end
function number.tocentimeters (n,fmt) return numbertodimen(n,"cm",fmt) end
function number.tomillimeters (n,fmt) return numbertodimen(n,"mm",fmt) end
function number.toscaledpoints(n,fmt) return numbertodimen(n,"sp",fmt) end
function number.toscaledpoints(n)     return            n .. "sp"      end
function number.tobasepoints  (n,fmt) return numbertodimen(n,"bp",fmt) end
function number.topicas       (n,fmt) return numbertodimen(n "pc",fmt) end
function number.todidots      (n,fmt) return numbertodimen(n,"dd",fmt) end
function number.tociceros     (n,fmt) return numbertodimen(n,"cc",fmt) end

\begin{tabular}{rl}
\textbf{Unit name} & \textbf{Description} \\\hline
bp & big point \\
cc & cicero \\
cm & centimeter \\
dd & didot \\
em & horizontal measure of \emph{M} \\
ex & vertical measure of \emph{x} \\
in & inch \\
mm & milimeter \\
nc & new cicero \\
nd & new didot \\
pc & pica \\
pt & point \\
sp & scaledpoint \\
\end{tabular}

\begin{multicols}{2}
\begin{minted}{latex}
\luakeysdebug{
  bp = 1bp,
  cc = 1cc,
  cm = 1cm,
  dd = 1dd,
  em = 1em,
  ex = 1ex,
  in = 1in,
  mm = 1mm,
  nc = 1nc,
  nd = 1nd,
  pc = 1pc,
  pt = 1pt,
  sp = 1sp,
}
\end{minted}
\begin{minted}{lua}
{
  ['bp'] = 65781,
  ['cc'] = 841489,
  ['cm'] = 1864679,
  ['dd'] = 70124,
  ['em'] = 655360,
  ['ex'] = 282460,
  ['in'] = 4736286,
  ['mm'] = 186467,
  ['nc'] = 839105,
  ['nd'] = 69925,
  ['pc'] = 786432,
  ['pt'] = 65536,
  ['sp'] = 1,
}
\end{minted}
\end{multicols}

%%
%
%%

\subsection{string}

\begin{multicols}{2}
\begin{minted}{latex}
\luakeysdebug{
  without quotes = no commas and equal signs are allowed,
  with double quotes = ", and = are allowed",
}
\end{minted}
\begin{minted}{lua}
{
  ['without quotes'] = 'no commas and equal signs are allowed',
  ['with double quotes'] = ', and = are allowed',
}
\end{minted}
\end{multicols}

%-----------------------------------------------------------------------
%
%-----------------------------------------------------------------------

\section{Exported functions of the Lua module \texttt{luakeys.lua}}

The Lua module exports this functions:

\begin{minted}{lua}
local luakeys = require('luakeys')
local parse = luakeys.parse
local render = luakeys.render
local print = luakeys.print
local stringify = luakeys.stringify
\end{minted}

%%
%
%%

\subsection{\texttt{parse(kv\_string, options)}: table}

The function \lua{parse(input_string, options)} is the main method of
this module. It parses a key-value string into a Lua table.

\begin{minted}{latex}
\newcommand{\mykeyvalcmd}[1][]{
  \directlua{
    result = luakeys.parse('#1')
    luakeys.print(result)
  }
  #2
}
\mykeyvalcmd[one=1]{test}
\end{minted}

\noindent
In plain \TeX:

\begin{minted}{latex}
\def\mykeyvalcommand#1{
  \directlua{
    result = luakeys.parse('#1')
    luakeys.print(result)
  }
}
\mykeyvalcmd{one=1}
\end{minted}

\noindent
The function can be called with a options table. This two options are
supported.

\begin{minted}{lua}
local result = parse('one,two,three', {
  convert_dimensions = false,
  unpack_single_array_value = false
})
\end{minted}

%%
%
%%

\subsection{\texttt{render(tbl)}: string}

The function \lua{render(tbl)} works in reverse of the function
\lua{parse(lua_table)}. It takes a Lua table and converts this table
into a key-value string. The resulting string usually has a different
order as your input table.

\begin{minted}{lua}
result = luakeys.parse('one=1,two=2,tree=3,')
print(luakeys.render(result))
--- one=1,two=2,tree=3,
--- or:
--- two=2,one=1,tree=3,
--- or:
--- ...
\end{minted}

In Lua, only tables with 1-based consecutive integer keys (a.k.a. array
tables) can be parsed in order.

\begin{minted}{lua}
result = luakeys.parse('one,two,three')
print(luakeys.render(result))
--- one,two,three, (always)
\end{minted}

%%
%
%%

\subsection{\texttt{print(tbl): void}}

The function \lua{print(tbl)} pretty prints a Lua table to stdout. It is
a utility function that can be used to debug and inspect the resulting
Lua table from the parse function. You have to use a console to compile
your \TeX{} document to see the terminal output.

\begin{minted}{md}
{
  ['level1'] = {
    ['level2'] = {
      ['level3'] = {
        ['str'] = 'lua',
        ['bool'] = true,
        ['num'] = -0.001,
        ['dim'] = 1864679,
      },
    },
  },
}
\end{minted}

%%
%
%%

\subsection{\texttt{stringify(tbl, for\_tex): string}}

The function \lua{stringify(tbl, for_tex)} converts a Lua table into a
printable string. This function is used to realize the print() function
from above. \lua{stringify(tbl, true)} (\lua{for_tex = true}) generates
a string which can be embeded into \TeX{}-documents. The macro
|\luakeysdebug{}| uses this option. \lua{stringify(tbl, false)} or
\lua{stringify(tbl)} generating a string suitable for the terminal.

%-----------------------------------------------------------------------
%
%-----------------------------------------------------------------------

\section{Implementation}

\inputminted{lua}{luakeys.lua}

\pagebreak
\PrintChanges
\pagebreak
\PrintIndex
\end{document}
