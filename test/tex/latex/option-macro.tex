\documentclass{article}
\usepackage{luakeys}

\begin{document}

\section{A macro name consists of letters only}

\directlua{
  local parser = luakeys.define({key = { macro = 'MyMacro' }})
  local result = parser('key=The content of a macro with a name containing only letters.')
}

\MyMacro

\section{A macro name with an @ character}

\directlua{
  local parser = luakeys.define({key = { macro = 'My@Macro' }})
  local result = parser('key=The content of a macro with an “at” symbol in it.')
}

\makeatletter
\My@Macro
\makeatother

\section{A macro that expands to numbers}

\directlua{
  local parser = luakeys.define({key = { macro = 'MyNumberMacro' }})
  local result = parser('key=123')
  tex.print(type(result.key))
}

\MyNumberMacro

\def\TestDifferentValueTypes#1{
\bigskip
Value: #1 \par Lua datatye:
\directlua{
  local parser = luakeys.define({key = { macro = 'MyMacro' }})
  local result = parser('key=#1')
  tex.print(type(result.key))
}

expands to: “\MyMacro”
}

\meaning\MyMacro

\TestDifferentValueTypes{1.23}

\TestDifferentValueTypes{A string}

\TestDifferentValueTypes{true}

\TestDifferentValueTypes{False}

\section{non-existent key}

\directlua{
  local parser = luakeys.define({key = { macro = 'MyMacro' }})
  local result = parser('unknown=value', { no_error = true })
}

a non-existent key produces a macro that has no content:
“\MyMacro”

\end{document}
